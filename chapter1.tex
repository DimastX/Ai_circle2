\section{Модель EIDM}
\label{sec:eidm_model}

Рассмотрим однополосную дорогу, на которой движутся автомобили. Для каждого автомобиля $i$ в момент времени $t$ заданы следующие характеристики:
\begin{itemize}
    \item координата $x_i(t)$,
    \item скорость $v_i(t)$,
    \item дистанция $s_i(t)$ до впереди идущего автомобиля.
\end{itemize}
\begin{figure}[h]
	\centering
	\includegraphics[width=0.8\linewidth]{EIDM.png}
	\caption{Схема движения автомобилей}
	\label{fig:EIDM}
\end{figure}

\subsection{Уравнение движения}
\label{subsec:eidm_equation}

Пусть $x_i(t)$ — координата автомобиля $i$, $v_i(t)=dx_i/dt$ — его скорость, а $s_i(t)= x_{i-1}(t)- x_i(t)- L$ — дистанция до лидирующего автомобиля, где $L$ — длина автомобиля. Определим относительную скорость как $\Delta v_i= v_i- v_{i-1}$.

Модель Enhanced Intelligent Driver Model (EIDM) \cite{treiber2013,kesting2010} описывает динамику движения автомобиля через уравнение ускорения:
\begin{equation}
\label{eq:eidm}
\frac{dv_i}{dt} = a \left[1 - \left(\frac{v_i}{v_0}\right)^\delta - \left( \frac{s^*(v_i, \Delta v_i)}{s_i} \right)^2 \right],
\end{equation}
где функция желаемой дистанции имеет вид:
\begin{equation}
\label{eq:s_star}
s^*(v_i, \Delta v_i)= s_0 + T \cdot v_i + \frac{v_i \cdot (-\Delta v_i)}{2 \sqrt{a b}}.
\end{equation}

Параметры модели имеют следующий физический смысл:
\begin{itemize}
    \item $a$ — комфортное ускорение,
    \item $b$ — комфортное торможение,
    \item $s_0$ — минимальный безопасный зазор,
    \item $T$ — желаемый временной интервал между автомобилями,
    \item $v_0$ — желаемая скорость движения,
    \item $\delta$ — показатель степени, определяющий форму функции ускорения (обычно $\delta=4$).
\end{itemize}

Физический смысл модели заключается в следующем: при низкой скорости ($v_i \ll v_0$) автомобиль интенсивно ускоряется, при приближении к желаемой скорости ($v_i \sim v_0$) интенсивность разгона уменьшается, а при сокращении дистанции ($s_i < s^*(...)$) возникает торможение. Подробное описание модели и её свойств приведено в работах \cite{treiber2013,helbing2001}.

\section{Равновесное состояние}
\label{sec:equilibrium}

\subsection{Условия стационарности}
\label{subsec:equilibrium_conditions}

В стационарном режиме движения выполняются следующие условия:
\begin{itemize}
    \item ускорение равно нулю: $dv_i/dt=0$,
    \item относительная скорость отсутствует: $\Delta v_i=0$.
\end{itemize}

Подставляя $dv_i/dt=0$ в уравнение EIDM (\ref{eq:eidm}), получаем:
\begin{equation}
\label{eq:equilibrium_eidm}
0 = a \left[1 - \left(\frac{v_e}{v_0}\right)^\delta - \left( \frac{s^*(v_e,0)}{s_e} \right)^2 \right].
\end{equation}

При $\Delta v=0$ функция желаемой дистанции упрощается: $s^*(v_e,0)= s_0+ T\cdot v_e$. Следовательно, равновесное состояние описывается уравнением:
\begin{equation}
\label{eq:equilibrium_final}
\left(\frac{v_e}{v_0}\right)^\delta + \left[ \frac{s_0+ T\cdot v_e}{s_e} \right]^2= 1.
\end{equation}

Здесь $v_e$ и $s_e$ — равновесные скорость и дистанция соответственно. В случае свободного потока выполняется приближение $s_e \approx s_0+ T\cdot v_e$.

\subsection{Интерпретация равновесного состояния}
\label{subsec:equilibrium_interpretation}

Равновесное состояние характеризуется балансом двух факторов:
\begin{itemize}
    \item Член $(v_e/v_0)^\delta$ отражает стремление водителя достичь желаемой скорости.
    \item Член $\left[ (s_0+ T\cdot v_e)/ s_e\right]^2$ характеризует влияние плотности потока на движение.
\end{itemize}

Если $v_e< v_0$, автомобиль стремится ускориться, но это стремление ограничивается необходимостью поддерживать безопасную дистанцию.

\section{ Линеаризация и анализ устойчивости (исправлено)}

Рассматривается функция ускорения
\[
F(v,s,\Delta v)=
a\Bigl[1-\Bigl(\frac{v}{v_0}\Bigr)^{\delta}-\Bigl(\frac{s^\ast(v,\Delta v)}{s}\Bigr)^{2}\Bigr],
\qquad
s^\ast(v,\Delta v)=s_0+T\,v+\frac{v(-\Delta v)}{2\sqrt{ab}} ,
\]
где $a,b,s_0,T,v_0,\delta$ – параметры модели.


Введём малые возмущения
\[
v_i(t)=v_e+\delta v_i(t), \qquad 
s_i(t)=s_e+\delta s_i(t), \qquad
\Delta v_i(t)=\delta v_i(t)-\delta v_{i-1}(t).
\]

\subsection{ Частные производные функции ускорения}

Обозначим 
\[
\varphi=\Bigl(\frac{v}{v_0}\Bigr)^{\delta}, \qquad 
\psi=\Bigl(\frac{s^\ast}{s}\Bigr)^{2}.
\]
Тогда $F=a(1-\varphi-\psi)$.

\paragraph{Производная по дистанции $s$.}
Так как $s^\ast$ не зависит от $s$, имеем
\[
\frac{\partial \psi}{\partial s}=
\frac{\partial }{\partial s}\frac{(s^\ast)^2}{s^{2}}
=-2\frac{(s^\ast)^2}{s^{3}},
\qquad
\boxed{%
	C=\Bigl.\frac{\partial F}{\partial s}\Bigr|_{e}
	=2a\,\frac{(s_0+Tv_e)^2}{s_e^{3}}}>0 \]

\paragraph{Производная по скорости $v$.}
\[
\frac{\partial \varphi}{\partial v}
=\frac{\delta}{v}\Bigl(\frac{v}{v_0}\Bigr)^{\delta}
=\frac{\delta}{v}\varphi.
\]
\[
\frac{\partial s^\ast}{\partial v}=T-\frac{\Delta v}{2\sqrt{ab}}
\;\xrightarrow{\;\Delta v=0\;}\;T .
\]
\[
\frac{\partial \psi}{\partial v}
=2\frac{s^\ast}{s^{2}}\frac{\partial s^\ast}{\partial v}
=2T\,\frac{s^\ast}{s^{2}} .
\]
Отсюда
\[
\boxed{%
A=\Bigl.\frac{\partial F}{\partial v}\Bigr|_{e}
=-a\Bigl[
\frac{\delta}{v_e}\Bigl(\frac{v_e}{v_0}\Bigr)^{\delta}
+2T\,\frac{s_0+Tv_e}{s_e^{2}}
\Bigr]}.
\]

\paragraph{Производная по относительной скорости $\Delta v$.}
\[
\frac{\partial s^\ast}{\partial\Delta v}=-\frac{v}{2\sqrt{ab}},\qquad
\frac{\partial \psi}{\partial \Delta v}
=2\frac{s^\ast}{s^{2}}\frac{\partial s^\ast}{\partial\Delta v}
=-\frac{v\,s^\ast}{\sqrt{ab}\,s^{2}} .
\]
Следовательно
\[
\boxed{%
B=\Bigl.\frac{\partial F}{\partial\Delta v}\Bigr|_{e}
=a\,\frac{v_e\,(s_0+Tv_e)}{\sqrt{ab}\,s_e^{2}} }.
\]

\subsection{ Линеаризованные уравнения}

Принимая
\[
\delta s_i=\delta x_{i-1}-\delta x_i,\qquad
\Delta v_i=\delta v_i-\delta v_{i-1},
\]
получаем систему
\[
\boxed{%
\dot{\delta v_i}=A\,\delta v_i
+B\,(\delta v_i-\delta v_{i-1})
+C\,\delta s_i },\qquad
\dot{\delta s_i}= \delta v_{i-1}-\delta v_i .
\]

\subsection{ Волновой анализ устойчивости}

Рассмотрим гармонические возмущения:
\[
\delta v_i = S\,e^{\lambda t} e^{i k i}, \qquad \delta x_i = R\,e^{\lambda t} e^{i k i}, \qquad k \in [-\pi, \pi].
\]
Подстановка даёт характеристическое уравнение:
\[
\boxed{
	\lambda^2 - e^{-\lambda \tau_r} \left[G(k)\lambda + H(k)\right] = 0,
}
\qquad\text{где}
\begin{cases}
	G(k) = A + B(1 - e^{-ik}),\\
	H(k) = C(e^{ik} - 1).
\end{cases}
\]

Наиболее критичная волна: $k = \pi$, тогда $e^{ik} = -1$, $1 - e^{-ik} = 2$,
и обозначим $D = A + 2B$. Тогда:
\[
G_\pi = D, \qquad H_\pi = -2C.
\]

Пусть на границе устойчивости $\lambda = i\omega$, где $\omega \in \mathbb{R}^+$.
Тогда вещественная и мнимая части:
\[
\begin{aligned}
	\omega D &= -\omega^2 \sin(\omega \tau_r),\\
	2C &= \omega^2 \cos(\omega \tau_r).
\end{aligned}
\]

Из второго уравнения:
\[
\omega^2 = \tfrac{1}{2} \left[ D^2 + \sqrt{D^4 + 16C^2} \right].
\]
Отсюда критическое значение времени реакции:
\[
\boxed{
	\tau_{\mathrm{cr}} = \frac{1}{\omega} \arccos\left(\frac{2C}{\omega^2}\right), \quad \omega \text{ — как выше}.
}
\]

\subsection{ Условия устойчивости}

Система устойчива тогда и только тогда, когда:
\[
\boxed{
	C > 0, \quad D = A + 2B < 0, \quad \tau_r < \tau_{\mathrm{cr}}.
}
\]

Для грубой оценки можно использовать приближение:
\[
\tau_r \lesssim \frac{|D|}{2C},
\]
которое даёт консервативную границу устойчивости.

\subsection*{Вывод}
Введение времени реакции существенно снижает устойчивость потока.
Даже при отрицательном $D$ и положительном $C$, избыточная задержка $\tau_r$ может вызвать возникновение колебаний типа stop-and-go. Расчёт критического значения $\tau_{\mathrm{cr}}$ позволяет оценить, насколько быстро должен реагировать водитель для поддержания стабильного режима движения.
